\documentclass[a5paper, 10pt]{article}

% Текст
\usepackage[utf8]{inputenc} % UTF-8 кодировка
\usepackage[russian]{babel} % Русский язык
\usepackage{indentfirst} % красная строка в первом параграфе в главе
% Отображение страниц
\usepackage{geometry} % размеры листа и отступов
\geometry{
	left=12mm,
	top=25mm,
	right=15mm,
	bottom=17mm,
	marginparsep=0mm,
	marginparwidth=0mm,
	headheight=10mm,
	headsep=7mm,
	nofoot}
\usepackage{afterpage,fancyhdr} % настройка колонтитулов
\pagestyle{fancy}
\fancypagestyle{style}{ % создание нового стиля style
	\fancyhf{} % очистка колонтитулов
	\fancyhead[LO, RE]{} % название документа наверху
	\fancyhead[RO, LE]{\leftmark} % название section наверху
	\fancyfoot[RO, LE]{\thepage} % номер страницы справа внизу на нечетных и слева внизу на четных
	\renewcommand{\headrulewidth}{0.25pt} % толщина линии сверху
	\renewcommand{\footrulewidth}{0pt} % толцина линии снизу
}
\fancypagestyle{plain}{ % создание нового стиля plain -- полностью пустого
	\fancyhf{}
	\renewcommand{\headrulewidth}{0pt}
}
\fancypagestyle{title}{ % создание нового стиля title -- для титульной страницы
	\fancyhf{}
	\fancyhead[C]{{\footnotesize
			Министерство образования и науки Российской Федерации\\
			Федеральное государственное автономное образовательное учреждение высшего образования
	}}
	\fancyfoot[C]{{\large 
			Санкт-Петербург, 2023
	}}
	\renewcommand{\headrulewidth}{0pt}
}

% Математика
\usepackage{amsmath, amsfonts, amssymb, amsthm} % Набор пакетов для математических текстов
%\usepackage{dmvnbase} % мехматовский пакет latex-сокращений
\usepackage{cancel} % зачеркивание для сокращений
% Рисунки и фигуры
\usepackage[pdftex]{graphicx} % вставка рисунков
\usepackage{wrapfig, subcaption} % вставка фигур, обтекая текст
\usepackage{caption} % для настройки подписей
\captionsetup{figurewithin=none,labelsep=period, font={small,it}} % настройка подписей к рисункам
% Рисование
\usepackage{tikz} % рисование
\usepackage{circuitikz}
\usepackage{pgfplots} % графики
% Таблицы
\usepackage{multirow} % объединение строк
\usepackage{multicol} % объединение столбцов
% Остальное
\usepackage[unicode, pdftex]{hyperref} % гиперссылки
\usepackage{enumitem} % нормальное оформление списков
\setlist{itemsep=0.15cm,topsep=0.15cm,parsep=1pt} % настройки списков
% Теоремы, леммы, определения...
\theoremstyle{definition}
\newtheorem{Def}{Определение}
\newtheorem*{Axiom}{Аксиома}
\theoremstyle{plain}
\newtheorem{Th}{Теорема}
\newtheorem{Lem}{Лемма}
\newtheorem{Cor}{Следствие}
\newtheorem{Ex}{Пример}
\theoremstyle{remark}
\newtheorem*{Note}{Замечание}
\newtheorem*{Solution}{Решение}
\newtheorem*{Proof}{Доказательство}
% Свои команды
\newcommand{\comb}[1]{\left[\hspace{-4pt}\begin{array}{l}#1\end{array}\right.\hspace{-5pt} } % совокупность уравнений
% Титульный лист
\usepackage{csvsimple-l3}
\newcommand*{\titlePage}{
	\thispagestyle{title}
	\begingroup
	\begin{center}
		%		{\footnotesize
			%			Министерство образования и науки Российской Федерации\\
			%			Федеральное государственное автономное образовательное учреждение высшего образования
			%		}
		%		
		\vspace*{6ex}
		
		{\small
			САНКТ-ПЕТЕРБУРГСКИЙ НАЦИОНАЛЬНЫЙ ИССЛЕДОВАТЕЛЬСКИЙ УНИВЕРСИТЕТ ИТМО	
		}
		
		\vspace*{2ex}
		
		{\normalsize
			Факультет систем управления и робототехники
		}
		
		\vspace*{15ex}
		
		{\Large \bfseries 
			Отчет по проектной работе
		}
\vspace*{2ex}
		
		{\Large \bfseries 
			по теме "Движение вращающегося мяча в воздухе с учетом эффекта Магнуса"
		}
\vspace*{2ex}
		
		{\Large
			по дисциплине "Основы классической механики"
		}

	\end{center}
	\vspace*{20ex}
	\begin{flushright}
		{\large 
			\underline{Выполнила}: студентка гр. \textbf{R3138}\\
			\begin{flushright}
				\textbf{Нечаева А.А.}\\
			\end{flushright}
		}
		
		\vspace*{5ex}
		
		{\large 
			\underline{Преподаватель}: \textit{Смирнов Александр Витальевич}
		}
	\end{flushright}	
	\newpage
	\setcounter{page}{2}
	\endgroup}

\begin{document}
	\titlePage
	\pagestyle{style}
\newpage
\section{Построение математической модели}
\subsection{Эффект Магнуса}
\textit{Эффект Магнуса} - физическое явление, возникающее при обтекании вращающегося тела потоком жидкости или газа. Возникающая сила - результат воздействия таких физических явлений, как \textit{эффект Бернулли} и образования \textit{пограничного слоя} в среде вокруг обтекаемого объекта, - действует на тело перпендикулярно направлению потока. Эффект описан немецким физиком \textit{Генрихом Магнусом} в 1853 году. \\\\
Вращающийся объект создает вокруг себя вихревое движение, с одной стороны направление вихря совпадает с направлением обтекающего потока, с другой - противоположно, следовательно, скорость движения среды с одной стороны повышается, с другой - уменьшается. \textit{Согласно уравнению Бернулл: чем меньше скорость, тем выше давление}. Возникающая разность давлений вызывает возникновение поперечной силы, вектор которой направлен от стороны, где направления вращения и потока противоположны, стороне с сонаправленными. Явление иллюстрирует рисунок 1.
\\\\
**рисунок 1**
\\\\
\subsection{Вывод уравнений и формул}	
Обозначим начальные условия полета мяча:\\
1. Мяч - \textit{абсолютно твёрдое тело}, то есть будем считать, что \textit{взаимное расположение точек мяча не меняется с течением времени}, деформацией во время полета пренебрежем\\
2. Известные параметры мяча: радиус (\textit{\textbf{R}}) и масса (\textit{\textbf{m}})\\
3. Также заданы начальные значения угловой ($\mathbf{\omega_{0}}$) и линейной скоростей ($\mathbf{v_{0}}$)\\
4. Задана некоторая плотность газа - среды, в которой происходит полет мяча ($\mathbf{\rho_{\text{г}}}$)\\\\

Запишем \textit{закон Ньютона} в общем виде
\begin{equation}
\vec{F} = m \cdot \vec{a}
\end{equation}
Далее распишем все силы, действующие на мяч в процессе полета
\begin{equation}
\vec{F}_{\text{тяжести}}+ \vec{F}_{\text{сопротивления среды}} + \vec{F}_{\text{Магнуса}}= m \cdot \vec{a}
\end{equation}
Выведем формулу для силы Магнуса, пусть $\Delta p $ - возникающая разность давлений на стороны шара, $S$ - площадь половины шара,  $k$ - коэффициент подъемной силы (\textit{для шара около 0.6}), тогда
\begin{equation}
F_{\text{Магнуса}}= \Delta p \cdot S\cdot k = 2 \pi \cdot R^{2} \cdot \Delta p \cdot k
\end{equation}
Применим следствие из уравнения Бернулли $\Delta p = \frac{\rho}{2} \left( v^{2}_{1}- v^{2}_{2} \right)$ и обозначим за $v$ линейную скорость тела относительно среды, $v_{\text{п}}$ - набегающего потока, получим
\begin{equation}
F_{\text{Магнуса}}= 2 \pi \cdot R^{2} \cdot \frac{\rho}{2} v v_{\text{п}} \cdot k = 0.6 \pi \rho  R^{2}  v v_{\text{п}}
\end{equation}
Запишем формулу для силы, создаваемой эффектом Магнуса в векторной форме
\begin{equation}
\vec{F}_{\text{Магнуса}}= 0.6 \pi \rho  R^{2} v v_{\text{п}} \cdot  \frac{   \vec{v}_{\text{п}} \times \vec{\omega}}{ v_{\text{п}} \cdot \omega} = 0.6 \pi \rho  R^{2} v  \cdot  \frac{   \vec{v}_{\text{п}} \times \vec{\omega}}{  \omega}
\end{equation}
Учтем действие силы сопротивления воздуха, применив формулу Стокса (\textit{коэффициент вязкости для воздуха при } $ T = 15 \text{ }C^{\circ} $ \textit{равен} $17.8 \textit{ мкПа}\cdot c $)
\begin{equation}
\vec{F}_{\text{сопротивления среды}} = - 6 \pi  \eta R \vec{v}
\end{equation}
Получим уравнение, описывающее движение мяча
\begin{equation}
m \vec{g} - 6 \pi  \eta R \vec{v} +0.6 \pi \rho  R^{2} v  \cdot  \frac{   \vec{v}_{\text{п}} \times \vec{\omega}}{  \omega} = m \vec{a}
\end{equation}
В общем случае будем рассматривать движение мяча в \textit{Декартовой системе координат}, в \textit{трехмерном пространстве}, тогда проекции на оси $X$, $Y$ и $Z$ соотвественно
\begin{multline}
m g_{x} - 6 \pi  \eta R v_{x} +0.6 \pi \rho  R^{2} v  \cdot  \frac{  \left( \vec{v}_{\text{п}} \times \vec{\omega} \right)_{x}}{  \omega} = m a_{x} \text{ , } v_{x} = v_0 sin \left( \alpha \right) \cdot cos  \left( \beta \right)
\end{multline}
\begin{multline}
m g_{y} - 6 \pi  \eta R v_{y} +0.6 \pi \rho  R^{2} v  \cdot  \frac{  \left( \vec{v}_{\text{п}} \times \vec{\omega} \right)_{y}}{  \omega} = m a_{y} \text{ , } v_{y} = v_0 sin \left( \alpha \right) \cdot sin  \left( \beta \right)
\end{multline}
\begin{multline}
m g_{z} - 6 \pi  \eta R v_{z} +0.6 \pi \rho  R^{2} v  \cdot  \frac{  \left( \vec{v}_{\text{п}} \times \vec{\omega} \right)_{z}}{  \omega} = m a_{z} \text{ , } v_{z} = v_0 cos \left( \alpha \right)
\end{multline}
Перейдем к уравнениям для угловой скорости вращения мяча $ \omega $, в частности, найдем выражения для разложения   $ \left( \vec{v}_{\text{п}} \times \vec{\omega} \right) $ по единичным векторам, задающим оси координат, $ \vec{i}$, $ \vec{j}$ и $ \vec{k}$
\begin{multline}
\left( \vec{v}_{\text{п}} \times \vec{\omega} \right) = 
\begin{vmatrix}
\vec{i} & \vec{j} & \vec{k} \\
v_{\text{п}_{x}} & v_{\text{п}_{y}} & v_{\text{п}_{z}} \\
\omega _{x} & \omega _{y} & \omega _{z}
\end{vmatrix}
= \vec{i} 
\begin{vmatrix}
 v_{\text{п}_{y}} & v_{\text{п}_{z}} \\
\omega _{y} & \omega _{z}
\end{vmatrix}
- \vec{j}
\begin{vmatrix}
 v_{\text{п}_{x}} & v_{\text{п}_{z}} \\
\omega _{x} & \omega _{z}
\end{vmatrix}
+ \vec{k}
\begin{vmatrix}
 v_{\text{п}_{x}} & v_{\text{п}_{y}} \\
\omega _{x} & \omega _{y}
\end{vmatrix}
=\\
= \vec{i}  \left( v_{\text{п}_{y}} \omega _{z} - \omega _{y} v_{\text{п}_{z}} \right) + \vec{j} \left( \omega _{x} v_{\text{п}_{z}} - v_{\text{п}_{x}} \omega _{z} \right) + \vec{k} \left(  v_{\text{п}_{x}} \omega _{y} - \omega _{x} v_{\text{п}_{y}} \right) \to \\
\to \left( \vec{v}_{\text{п}} \times \vec{\omega} \right)_{x} = v_{\text{п}_{y}} \omega _{z} - \omega _{y} v_{\text{п}_{z}}\\
 \left( \vec{v}_{\text{п}} \times \vec{\omega} \right)_{y} = \omega _{x} v_{\text{п}_{z}} - v_{\text{п}_{x}} \omega _{z}\\
 \left( \vec{v}_{\text{п}} \times \vec{\omega} \right)_{z} =  v_{\text{п}_{x}} \omega _{y} - \omega _{x} v_{\text{п}_{y}}\\
\end{multline}
Перейдем к дифференциальному виду уравнений проекций движения мяча на оси координат
\begin{equation}
m g_{x} - 6 \pi  \eta R x' +0.6 \pi \rho  R^{2} v  \cdot  \frac{ v_{\text{п}_{y}} \omega _{z} - \omega _{y} v_{\text{п}_{z}}}{  \sqrt{\omega _{x}^2+\omega _{y}^2+\omega _{z}^2}} = m x''
\end{equation}
\begin{equation}
m g_{y} - 6 \pi  \eta R y' +0.6 \pi \rho  R^{2} v  \cdot  \frac{ \omega _{x} v_{\text{п}_{z}} - v_{\text{п}_{x}} \omega _{z}}{ \sqrt{\omega _{x}^2+\omega _{y}^2+\omega _{z}^2}} = m y''
\end{equation}
\begin{equation}
m g_{z} - 6 \pi  \eta R z' +0.6 \pi \rho  R^{2} v  \cdot  \frac{v_{\text{п}_{x}} \omega _{y} - \omega _{x} v_{\text{п}_{y}} }{ \sqrt{\omega _{x}^2+\omega _{y}^2+\omega _{z}^2}} = m z''
\end{equation}
Запишем также начальные условия задачи, координаты мяча -- положение его центра масс во времени
\begin{equation}
\begin{cases}
\left(x (0), y(0), z(0) \right)  = \left(0, 0, 0\right)\\
v_{x} (0) = v_{0x}\\
v_{y} (0) = v_{0y}\\
v_{z} (0) = v_{0z}\\
\omega_{x} (0) = \omega_{0x}\\
\omega_{y} (0) = \omega_{0y}\\
\omega_{z} (0) = \omega_{0z}
\end{cases}
\end{equation}
\subsection{Решение дифференциальных уравнений}
Найдем зависимости $x(t)$, $y(t)$, $z(t)$, решив соотвествующую систему дифференциальных уравнений.\\
Будем считать, что ветра нет, поэтому скорость потока противоположна по направлению изменению линейной координаты мяча вдоль соответствующей оси $v_{\text{п}_{i}} = -i'$, также $g_z = g$, $g_x = 0$, $g_y = 0$, а вектор угловой скорости направлен вдоль оси $Z$, значит,  $\omega _{x} = 0$, $\omega _{y} = 0$ и $\omega _{z} = \omega$.
\begin{equation}
\begin{cases}
 - 6 \pi  \eta R x' - 0.6 \pi \rho  R^{2} v  \cdot  y' = m x''\\
 - 6 \pi  \eta R y' +0.6 \pi \rho  R^{2} v  \cdot   x'  = m y''\\
m g - 6 \pi  \eta R z'  = m z''
\end{cases}
\end{equation}
Уравнение относительно $z(t)$ решим как обычное линейное дифференциальное уравнение второго порядка.\\
Для удобства проведем замены $$6 \pi \eta R \to b$$ $$  mg  \to f$$
\begin{equation*}
m z'' + b z'-f = 0
\end{equation*}
\textit{Решим соотвествующее однородное уравнение и будем искать решения в виде } $z = e^{kt}$
\begin{equation*}
mk^{2}e^{kt} + bke^{kt} = 0
\end{equation*}
\begin{equation*}
mk^{2} + bk = 0
\end{equation*}
\begin{equation*}
k_1 = 0, k_2 = -\frac{b}{m}
\end{equation*}
\textit{Тогда общее решение } $\bar{z} = C_1 e^{-\frac{bt}{m}} + C_2$\\
\textit{Теперь найдем частное решение в виде } $z = At$
\begin{equation*}
bA = f
\end{equation*}
\begin{equation*}
A = \frac{f}{b}
\end{equation*}
\begin{equation}
\tilde{z} = \frac{ft}{b}
\end{equation}
В итоге получаем выражение $z(t) = \bar{z} + \tilde{z} =  C_1 e^{-\frac{bt}{m}} + C_2 + \frac{ft}{b}$ \\
И, возвращаясь к исходным коэффициентам
\begin{equation}
z(t) = C_1 e^{-\frac{6 \pi \eta Rt}{m}} + C_2 + \frac{mgt}{6 \pi \eta R}
\end{equation}

Давайте не будем учитывать сопротивление воздуха
\begin{equation}
\begin{cases}
  - 0.6 \pi \rho  R^{2} v  \cdot  y' = m x''\\
    0.6 \pi \rho  R^{2} v  \cdot   x'  = m y''\\
m g  = m z''
\end{cases}
\end{equation}
Тогда элементарно $ z = \frac{gt^2}{2} $. А решение системы дифференциальных уравнений найдем, выразив поочередно  $x$ через $y$ и соотвественно наоборот.
\begin{equation}
\begin{cases}
  - 0.6 \pi \rho  R^{2} v  \cdot  y' = m x''\\
    0.6 \pi \rho  R^{2} v  \cdot   x'  = m y''
\end{cases}
\to
\begin{cases}
    y' = - \frac{m x''}{ 0.6 \pi \rho  R^{2} v}\\
    \left(0.6 \pi \rho  R^{2} v \right)^2 \cdot   x'  = -m^2x''
\end{cases}
\end{equation}
Отсюда получаем
\begin{equation}
x(t) = C_2 \sin \left(\frac{0.6 \pi \rho  R^{2} v \cdot t}{m} \right) +  C_1 \cos \left(\frac{0.6 \pi \rho  R^{2} v \cdot t}{m} \right) + C
\end{equation}
Аналогичное выражение получаем для $y$
\begin{equation}
y(t) = C_2 \sin \left(\frac{0.6 \pi \rho  R^{2} v \cdot t}{m} \right) +  C_1 \cos \left(\frac{0.6 \pi \rho  R^{2} v \cdot t}{m} \right) + C
\end{equation}
\end{document}













